\documentclass[conference]{IEEEtran}
\IEEEoverridecommandlockouts
% The preceding line is only needed to identify funding in the first footnote. If that is unneeded, please comment it out.
\usepackage{cite}
\usepackage{amsmath,amssymb,amsfonts}
\usepackage{algorithmic}
\usepackage{graphicx}
\usepackage{textcomp}
\usepackage{xcolor}
\usepackage[back-end=biber, style=ieee]{biblatex}
\addbibresource{refs.bib}
\begin{document}

\title{IoT Traffic Management System}

\author{\IEEEauthorblockN{Rohit Singh}
\IEEEauthorblockA{\textit{Electrical Engineering} \\
\textit{University of Waterloo}\\
Waterloo, Ontario \\
rkochhar@uwaterloo.ca}
\and
\IEEEauthorblockN{Vojdan Bojcev}
\IEEEauthorblockA{\textit{Electrical Engineering} \\
\textit{University of Waterloo}\\
Waterloo, Ontario \\
vbojcev@uwaterloo.ca}
}

\maketitle

\begin{abstract}
This document is a model and instructions for \LaTeX.
This and the IEEEtran.cls file define the components of your paper [title, text, heads, etc.]. *CRITICAL: Do Not Use Symbols, Special Characters, Footnotes, 
or Math in Paper Title or Abstract. \cite{Chong}\cite{Rizwan}\cite{Avetifipour}\cite{GOTTLICH}\cite{Ghena}.
\end{abstract}

\section{Introduction}
\section{Literature Review}
\section{Experiment Methodology}

\subsection{Modelling of Transport Networks}

For our experiment, we need to create a model that can represent traffic flow between intersections. To accomplish this, we use a directed graph with intersections represented by stateful nodes and traffic flow represented by edges. The states of a given node are determined by the quantities held by neighbouring edges.

\section{Experimental Results}


\printbibliography
\end{document}

\iffalse %NOTES/BRAINSTORMING

CHONG:

Discuss different network topologies: on-road, on-vehicle, or hybrid sensors. Discuss viability of having several wireless sensors at each intersection with one base station per intersection. Base stations can either be subservient to a master station or compute the optimization in a distributed manner. 

Goal of optimization is to minimize unused green-light time. Optimization is based on local load (single-intersection) but also must consider downstream (at least adjacent) intersections as well. Flat-hierarchy/local computation architectures could be better for local optimization where only 1-hop adjacent intersections are considered whereas multi-leveled/globally-computed architectures could be better for city-wide computations.

Generally, traffic algorithms are designed assuming a perfect network (i.e established city wifi/ethernet) which means state knowledge can be always known and known globally. Our project would take into consideration IoT devices that have their own network and low-power or self-powered devices such that the traffice network is minimally dependent on existing architecture (could be applicable to high-traffic, underdeveloped infrastructure locations; For example, much of south and south-east Asia). Discuss the different WSN protocols mentioned in class and explore their viabilisty in areas like range, power requirements, security risks.

RIZWAN:



\endif